\section{First period}\label{first_period}
Evaluation for the first period of the project. For a summary and the
evaluation go directly to section \ref{first_period_evaluation}.

\subsection{Week 1 - 2 September}\label{week_1}
The project started with two days off for me while participating in the \"{O}
till \"{O} competition in Stockholm Archipelago \cite{web:otillo}.

Wednesday consisted of reading up on Project VASA and I got an introduction
from Simon about the project. I was put in the Software group. The software
part was divided into four parts where I (together with Joel) got responsiblity
for the "Mission Control System".

Thursday and Friday consisted of collecting articles about High-Level
Architecture and Mission Control System. Together with Joel a first definition
of what the Mission Control is and should do was defined.

\subsection{Week 2 - 9 September}\label{week_2}
The beginning of the week consisted of writing the "High-Level Architecture"
and "Mission Control" part in the State of the Art document. I also wrote
the method that I had used to gather the research articles which was later
removed from the State of the Art document.

Time was also spent on quite a few discussions ranging from how to structure
the State of Art document to general brainstorming about the AUV. A group wide
discussion took place about how all components were to be connected to
each other.

As responsibility on the side I have the website and I have also helped Daniel
with LaTeX quite a lot. Wednesday and thursday mainly consisted of proofreading
and fixing several LaTeX and bibtex problems, as well as fixing my own parts
when I got feedback from other project members.

After finishing the first draft of the State of the Art document on friday
morning I started to try Ada out.

\subsection{Week 3 - 16 September}\label{week_3}
Monday. After finishing the first draft of State of the Art, time
had come to go back to our requirements list to make sure we didn't miss
anything. One part that was missing was the "Space plug-and-play avionics"
part for software "above" firmware. I also started to look for alternatives
to the RaspberryPI as mainboard, several was found.

Tuesday. Time had come to setup the website so as I've experience with
wordpress I decided for that and found a suitable free theme that we could
start from. Set up staging server as local guest system but ran into several
problems due to failure with eduroam and the network for guest systems. I also
helped Joel with reinstalling Ubuntu.

Wednesday. Got new information during the tuesday about the plan for the entire
project and information about the self-evaluation so started with the
evaluation for the first weeks. The rest of the day mostly concerned setting
up the infrastructure in the project room such as reinstalling "mothership"
from earlier years and aquiring access to the router laying around here.

The "Mothership" computer will be used as build server and temporary host
our website. Public and static ip requested from Helpdesk. Some reading
in the end of the day about Space plug-and-play avionics.

Thursday. Started the day with an infrastructure sketch, how to set up the
commit and build flow for developers and the build server. Wrote the
state of the art part about mainboard choice. Later during the day I started
to look into SPA a bit deeper.

Friday. Started the day with reading a little bit more about the BeagleBone
Black board as well as Arduino Due/Yun. Still confident with BeagleBone Black
choice as the best one though not sure if it will run all the software we
require. Continued with proofreading the State of the Art about mainboard
choice.

\subsection{Week 4 - 23 September}\label{week_4}
\subsection{Evaluation}\label{first_period_evaluation}
\subsubsection{A lot of small parts ends out taking a lot of time}
Helping out with LaTeX and Ubuntu where I can as well as setting up and
mainting the website.

\subsubsection{What is what?}
What is "Mission Control"? What is "Navigation"? What is "Motion Control"?
