\section{Introduction}
This report is a self evaluation during project "Naiad" in the course
"DVA425 - Project in Advanced Embedded Systems" at M\"{a}lardalens University.
This self evaluation looks back on the first 5 weeks of the project.

I'm a member of the software group.
% The report should be typeset with \LaTeX !
% The requirements that are set for the different grades are:
% \begin{description}
% \item[Grade 3] Be there and do a good job
% \item[Grade 4] Be there and take initiatives to fulfill the goals
% \item[Grade 5] Be there, take initiatives to fulfill the goals and suggest (close to research) new approaches
% \end{description}

\section{My contribution during this period}
\subsection{State of the Art}
For the State of the Art research I was responsible for the High-level
architecture and Mission Control (as subparts of the Mission Control System).
For both topics a few good papers were found
\cite{palomeras2012,palomeras2011,martinez2008,martinez2013}.
Especially the Ph.D. Dissertation from Palomeras \cite{palomeras2011}.

\subsubsection{High-level Architecture}
From the literature review I've given the suggestion to try out
Real-time Container Component Model (RT-CCM),
Lightweight CORBA Component Model (LwCCM) and
Component Oriented Layer-based Architecture for Autonomy (COLA2) to see if
these already available component-based models meet our
customer requirement on a component based implementation.

\subsubsection{Mission Control}
For the Mission Control my suggestion is to first create a simple domain
specific language in XML or basic scripts for faster prototyping. A simple
GUI could also be created to speed up the process of creating different missions.
To make the system more fault tolerant the mission control part
should be structured with well defined primitives (defined by Palomeras
\cite{palomeras2011}) and may use Petri Nets for this.

\subsubsection{Research Methodology}
During the writing of the "State of the Art" document I suggested a structured
research methodology (a systematical literature review) on how to find the relevant
papers for our different topics. Though after a few discussions this
suggestion was discarded due to time constraints.

\subsection{Infrastructure}
Following the first literature review period my main focus has been on setting
up infrastructure such as WLAN, build server and web server.
I have also registered the domain naiad.se. Priority has been
on documenting this process and making it as open as possible so others can do
the same thing with limited experience.
A limiting factor has been the lack of public ip address which has
made the infrastructure quite a lot more complex than first anticipated.
Per-Erik wrote the main part for the build script while I set up the Jenkins
CI server, Apache2 http server, syncing between Github, local build server and our web
server.

\subsection{State of the Art - more responsibility}
After the project leaders recieved the first feedback from our State of the Art
draft I was given responsibility of choosing mainboard as well as reading up on
Space Plug-and-Play Avionics (SPA).

\subsubsection{Mainboard}
From the beginning the goal for the project was a distributed system with a
small credit card sized computer as mainboard instead of a powerful mini-ITX
system. During the initial State of the Art work everyone assumed that
Raspberry PI was the best choice. Though after a few days I came to the
conclusion that the BeagleBone Black was a better choice for our project.
The BeagleBone Black is open hardware so will hopefully be much easier to
interact with than if we would have gone with the Raspberry PI.

\subsubsection{Space plug-and-play Avionics and CAN bus}
Space plug-and-play avionics was the last responsbility I got, others had briefly
looked in to it. Implementing Space plug-and-play avionics
over the CAN bus might show some difficulties. The first problem to solve is
the assignment of message IDs used within the CAN bus network for each connected
node. The CAN bus specification \cite{web:canspec} says that this should be pre-defined but with
SPA it most likely have to be decided by some kind of "SPA Manager (SM-x)". As of
writing this self evaluation, reading up on SPA and CAN bus is ongoing. The
outcome of this will most likely change priorities on what to work on when it
comes to the "High-level Architecture".

\section{Related work}
\subsection{Documentation}
With my experience from earlier reports and LaTeX I've been able to help Daniel
quite a lot with documentation issues such as the choice of LaTeX template
(ACM template used), fixing bibliography errors and helping
others with LaTeX. I've also helped out with suggesting on how to organise
future documentation.

\subsection{Git and Github}
With over 18 people working on the same project I suggested a move to Git and
Github as version control system for all of our code. Git makes merging
different branches easier and Github gives a visual presentation of our
repositories. Another benefit of Github is that it's a social network for
developers which may make it easier for other developers to get in contact
with our project. At the moment there is one repository for hardware, electronics
and software. This will probably change a little during the development cycle
when everyone starts to write code.

\section{Suggested grade \& motivation}
Suggested grade 5.

I've contributed with my knowledge in research methods, report writing and LaTeX
which have made the state of the art writing process run smooth. When it comes to
content in high-level architecture and mission control I've suggested
different solutions which are the current state of the art in research.

With regard to the infrastructure; the web server, build server and code repositories
are set up in such a way that it will be easy to maintain and keep up to date
throughout the project. The use of continous integration to check for regressions
will help the project keep the codebase buildable and detect bugs early on.

Continued research on the Space plug-and-play avionics system will hopefully give
the project a solution that makes the system easy to assembly.
