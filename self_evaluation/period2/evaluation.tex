\section{Introduction}
This report is a self evaluation during project "Naiad" in the course
"DVA425 - Project in Advanced Embedded Systems" at M\"{a}lardalens University.
This self evaluation looks back on week 5 to 10 of the project.

I'm a member of the software group.
% The report should be typeset with \LaTeX !
% The requirements that are set for the different grades are:
% \begin{description}
% \item[Grade 3] Be there and do a good job
% \item[Grade 4] Be there and take initiatives to fulfill the goals
% \item[Grade 5] Be there, take initiatives to fulfill the goals and suggest (close to research) new approaches
% \end{description}

\section{My contribution during this period}

\subsection{Virtual Network Protocol}
With external help I was able to go forward with Space plug-and-play avionics.
As most of that is under USA export regulations a new initiative has been
taken to create a public de facto standard that is called Virtual Network Protocol (VNP)
around the same principles by Fredrik Bruhn and MDH.

In the beginning of this period focus was on IP over CAN and solving the
Address Resolution (ARP) issue with the CAN Bus. Related work found were
the IETF Draft from 2001 \cite{ietf:ip_over_can}, "Porting the Internet Protocol
to the Controller Area Network" \cite{web:porting_ip_can} and "IP over CAN,
Transparent Vehicular to Infrastructure Access" \cite{lindgren2008}.

After a few days I reprioritised the work
together with project lead. Work has since then been put into a design
of VNP in the Naiad project \cite{web:naiad_auv_vnp, web:naiad_auv_vnp_design}.
During the implementation I have hit some obstacles with Ada which has lead to
not meeting the initial plan.

\subsection{Using tasks with Ada and Ravenscar}
As a part of the Virtual Network Protocol implementation I was challenged with
the problem of how to properly create a main loop for the applications that
will run on the BeagleBone Blacks. Another problem from the beginning was that each
VNP application should share information with other applications through
sockets. So the basic requirements I had gathered was that the main loop should
support listening on sockets, sending on sockets as well as doing "real work".

I decided upon using tasks in Ada as well
as keeping an eye on the restrictions the Ravenscar-profile defines on tasks.
Sources used during this process was "Extending Ravenscar with CSP Channels"
\cite{atiya2005} and "Guide for the use of the Ada Ravenscar Profile in
High-Integrity Systems" \cite{burns2003}, as well as various online sources
about game loops.

A simple example with Ada Tasks and TCP sockets has been created as a result of
this work.

Clarification from the customer about two project requirements during the last
week of period 2 has made it clear that sockets should not be used.
For the software running on the BeagleBone Blacks priority will be on Ada Tasks
with the Ravenscar profile using Protected Objects (shared memory) for
inter-process communication.

\section{Related work}
\subsection{Documentation}
Early on in this period I created a template in LaTeX for our final report
that everyone can use. All subparts within the project such as the Mission Control System,
Navigation and so forth will be in their own report with abstract, introduction,
conclusions etc. The complete Naiad report will then be a merge of all the PDF files.
This structure will in the end make it easy for the reader to read the
specific part he or she is interested in.

\subsection{Git and Github}
Continued to help other project members with Git and Github problems.

\subsection{Mainboard}
The BeagleBone Blacks for the Mission Control System and Sensor Fusion
has arrived and Ubuntu has been installed on one of them. Basic test compilation
has been made with GNAT to compile Ada code for the ARM architecture.

\section{Suggested grade \& motivation}
Suggested grade X.

I've suggested a design for VNP within the Naiad project which is work close to
reasearch as well as tried to implement the basic building blocks of that
design.

My work with IP over CAN is as well close to research. IP over CAN has been
introduced by a few research groups in the last decade. The main problem here
is the contradiction between IP as an address protocol and CAN Bus as a message
based protocol.
