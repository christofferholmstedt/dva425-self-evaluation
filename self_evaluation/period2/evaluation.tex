\section{Introduction}
This report is a self evaluation during project "Naiad" in the course
"DVA425 - Project in Advanced Embedded Systems" at M\"{a}lardalens University.
This self evaluation looks back on week 5 to 10 of the project.

I'm a member of the software group.
% The report should be typeset with \LaTeX !
% The requirements that are set for the different grades are:
% \begin{description}
% \item[Grade 3] Be there and do a good job
% \item[Grade 4] Be there and take initiatives to fulfill the goals
% \item[Grade 5] Be there, take initiatives to fulfill the goals and suggest (close to research) new approaches
% \end{description}

\section{My contribution during this period}
% \subsubsection{Mission Control}
% For the Mission Control my suggestion is to first create a simple domain
% specific language in XML or basic scripts for faster prototyping. A simple
% GUI could also be created to speed up the process of creating different missions.
% To make the system more fault tolerant the mission control part
% should be structured with well defined primitives (defined by Palomeras
% \cite{palomeras2011}) and may use Petri Nets for this.

\subsection{Virtual Network Protocol}
With external help I was able to go forward with Space plug-and-play avionics.
As most of that is under american export regulations a new initiative has been
taken to create a public standard that is called Virtual Network Protocol (VNP)
around the same principles.

The contribution made during this part of the project has been focused on
solving the Address Resolution (ARP) issues with the CAN Bus as well as creating a minimal implementation
in Ada.

\subsection{Using tasks with Ada and Ravenscar}
As a part of the Virtual Network Protocol implementation I was challenged with
the problem about how to properly create a main loop for the applications that
will run on the BeagleBone Blacks. All applications will listen on at least one
network (socket) as well as doing work while waiting for new input so the applications
can't just wait for network events.

I looked into this problem and decided upon using tasks in Ada as well as keeping
an eye on the restrictions Ravenscar has on tasks in Ada. Sources used during
this process was "Extending Ravenscar with CSP Channels" \cite{atiya2005}, TODO
more sources.
% Space plug-and-play avionics was the last responsbility I got, others had briefly
% looked in to it. Implementing Space plug-and-play avionics
% over the CAN bus might show some difficulties. The first problem to solve is
% the assignment of message IDs used within the CAN bus network for each connected
% node. The CAN bus specification \cite{web:canspec} says that this should be pre-defined but with
% SPA it most likely have to be decided by some kind of "SPA Manager (SM-x)". As of
% writing this self evaluation, reading up on SPA and CAN bus is ongoing. The
% outcome of this will most likely change priorities on what to work on when it
% comes to the "High-level Architecture".

\section{Related work}
\subsection{Documentation}
Early on in this period I created a template in LaTeX for our final report that everyone can
use. All subparts within the project such as the Mission Control System,
Navigation and so forth will be in their own report with abstract, introduction,
conclusions etc. The complete Naiad report will then be a merge of all the PDF files.
This structure will in the end make it easy for the reader to read the
specific part she is interested in.

\subsection{Infrastructure}
After some reorganisation of the code early on in the period the build scripts
have had to be updated. TODO

\subsection{Git and Github}
Continued to help other project members with Git and Github problems.

\subsection{Mainboard}
The BeagleBone Blacks for the Mission Control System and Sensor Fusion part
has arrived and Ubuntu has been installed on them as well as testing of GNAT
for ARM.

\section{Suggested grade \& motivation}
Suggested grade TODO XXX.
