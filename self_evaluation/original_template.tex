% !TEX TS-program = pdflatex
% !TEX encoding = UTF-8 Unicode


\documentclass[11pt,a4paper]{amsart}
%\geometry{landscape}                % Activate for for rotated page geometry
%\usepackage[parfill]{parskip}    % Activate to begin paragraphs with an empty line rather than an indent
\usepackage[T1]{fontenc}
\usepackage[utf8]{inputenc} % set input encoding (not needed with XeLaTeX)
\usepackage{graphicx}
\usepackage{amssymb}
\usepackage{epstopdf}
\usepackage{fancyhdr}
\usepackage[pdfborder={0 0 0},colorlinks=true, urlcolor=blue, citecolor=red, bookmarks=false]{hyperref}

	% Page style
	\pagestyle{fancy}

\DeclareGraphicsRule{.tif}{png}{.png}{`convert #1 `dirname #1`/`basename #1 .tif`.png}


\title{Self evaluation in CDT508/DVA410}
\author{NN}

\date{}                                           % Activate to display a given date or no date

\begin{document}
\maketitle
\thispagestyle{fancy}
\fancyhead[R]{Electronics/mechanics/.., Vasa/Void}
\fancyhead[L]{\today, NN}
\section{Introduction}
The report should be typeset with \LaTeX !
The requirements that are set for the different grades are:
\begin{description}
\item[Grade 3] Be there and do a good job
\item[Grade 4] Be there and take initiatives to fulfill the goals
\item[Grade 5] Be there, take initiatives to fulfill the goals and suggest (close to research) new approaches
\end{description}

\section{My contribution during this period}
Be clear on this work reflects on the grading criteria, i.e reading scientific papers~\cite{Ekstrom20011845221} or technical reports~\cite{thruster}
\section{How has my work contributed to the project}

\section{Related work[optional]}
Other work that supports the validity of my contribution~\cite{Ekstrom20011845221}.\\

\section{Suggested grade \& motivation}


\begin{thebibliography}{99}
\bibitem{Ekstrom20011845221}{{Ekstr{\"o}m, M. and Hartmann, O. and Karlsson, E. and Lidstr{\"o}m, E. and Granberg, P. and Nygren, M.}, {Physical Review B - Condensed Matter and Materials Physics}, \textbf{Vol. 64 nr. 18, 2001},\textit{Antiferromagnetism in {Zn-doped La$_2$CuO$_4$} as observed by muon spin resonance spectroscopy}}
\bibitem{thruster}
Optimal Thruster Configuration for Omni-directional Underwater Vehicles - \url{http://ieeexplore.ieee.org/stamp/stamp.jsp?tp=&arnumber=724320}
\end{thebibliography}


%\subsection{}



\end{document}  